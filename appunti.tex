\documentclass{article}
\usepackage[utf8]{inputenc}
\usepackage{amsmath,amssymb}
\usepackage{graphicx}
\usepackage{geometry}
\usepackage{imakeidx}
\geometry{a4paper, top=2cm, bottom=2cm, left=2cm, right=2cm, heightrounded, bindingoffset=5mm}

\renewcommand\arraystretch{2}

\title{Appunti di Metodi Variazionali}
\author{Matteo Scarcella}
\date{Maggio 2024}

\AtBeginDocument{\renewcommand\contentsname{Indice}}

\begin{document}

\maketitle
\tableofcontents

\newpage
\section{Introduzione}
Questi appunti sono soltanto un riordionamento dei risultati esposti nel libro di "Metodi variazionali per il controllo ottimo" di Bruni, Di Pillo. 
\section{Spazi normati e minimi di funzionali}
Sia $\mathcal{Z} = \mathbb{R}^\nu$ lo spazio ambiente, con $z \in \mathcal{Z}$. Sia $\mathcal{D} \subset \mathcal{Z}$ un sotto-insieme dello spazio ambiente detto insieme ammissibile, se necessario descritto da $\mu < \nu$\footnote{Nel caso $\mu = \nu$ si avrebbe un insieme di vincoli finito e verrebbe meno il concetto di minimizzazione. Ad esempio se $\mu = \nu = 2$, i due vincoli potrebbero essere due rette che si intersecano in un punto, e tale punto risulterebbe l'unico punto ammissibile} vincoli di uguaglianza $h(z) = 0$ e $\sigma$ vincoli di disuguaglianza $g(z) \leq 0$, dove con $\sigma_a$ e $g_a(z)$ si fa riferimento ai soli vincoli di disuguaglianza attivi, ovvero verificati all'uguaglianza. Sia $[t_i,t_f]$ un intervallo di tempo. Siano
\begin{equation}
    \|z\| = \sup_{t\in[t_i,t_f]} \|z(t)\|
\end{equation}
\begin{equation}
    \|z\| = \sup_{t\in[t_i,t_f]} \|z(t)\| + \sup_t \|\dot{z}(t)\|
\end{equation}  
Rispettivamente norma forte e norma debole, scelte a priori sullo spazio ambiente. Sia $J : \mathcal{Z} \to \mathbb{R}$ un funzionale di costo. L'obiettivo è trovare il controllo che permetta di minimizzare il valore del funzionale $J$, ovvero risolvere il problema
\begin{equation}
    \begin{cases}
        \min J(z) \\
        z \in \mathcal{D}
    \end{cases}
\end{equation}
\nobreak 
\hfill
$\square$
\newline
\textbf{\textit{Definizione 1.29 -}} $z^*$ è un punto di minimo locale (forte o debole, in base alla norma scelta) se vale
\begin{equation}
J(z^*) \leq J(z) \quad \forall z \in \mathcal{D} \cap \mathcal{S}(z^*,\varepsilon)
\end{equation}
\nobreak 
\hfill
$\square$
\newline
\textbf{\textit{Definizione 2.5 -}} Definita la matrice Jacobiana dei vincoli attivi in un punto ammissibile $\overline{z}$
\begin{equation}
    \frac{\partial{(h,g_a)}}{\partial{z}} \bigg\rvert_{\bar{z}} = 
    \begin{pmatrix}
        \frac{dh}{dz} \\ 
        \frac{dg_a}{dz}
    \end{pmatrix}_{\bar{z}}
\end{equation} 
Allora $\bar{z}$ si dice punto di regolarità dei vincoli se la matrice Jacobiana dei vincoli attivi ha rango pieno, ovvero se
\begin{equation}
    rank \bigg\{ \frac{\partial{(h,g_a)}}{\partial{z}} \bigg\rvert_{\bar{z}} \bigg\} = \mu + \sigma_a
\end{equation}
\nobreak 
\hfill
$\square$
\newline
\textbf{\textit{Definizione 2.6 -}} Si definisce la funzione Lagrangiana
\begin{equation}
    L(z,\lambda_0,\lambda,\eta) = \lambda_0 J(z) + \lambda^T h(z) + \eta^T g(z)
\end{equation} 
Con $\lambda_0,\lambda,\eta$ moltiplicatori opportuni
\nobreak 
\hfill
$\square$
\medskip
\newline
\textbf{\textit{Teorema 2.7 - (Condizioni necessarie di minimo)}} In riferimento al problema di minimo vincolato con opportuni vincoli $h(z)$ e $g(z)$, sia $z^*$ un punto di minimo locale. Allora esistono moltiplicatori $\lambda_0^*,\lambda^*,\eta^*$ non tutti simultaneamente nulli tali che:
\begin{equation}
    \begin{cases}
        \frac{\partial L}{\partial z}\big\rvert^* = 0^T \\
        \eta_i^*g_i(z^*) = 0 \quad i = 1,2,\dots,\sigma_a \\
        \lambda_0^* \geq 0 \\
        \eta_i^* \geq 0  \quad i = 1,2,\dots,\sigma_a
    \end{cases}
\end{equation} 
\nobreak 
\hfill
$\square$
\newpage
\noindent
\textbf{\textit{Teorema 2.8 - (Condizioni di Kuhn-Tucker)}} In riferimento al problema di minimo vincolato con opportuni vincoli $h(z)$ e $g(z)$, sia $z^*$ un punto di minimo locale e di regolarità dei vincoli. Allora valgono le stesse condizioni del teorema (2.7), con l'aggiunta di
\begin{equation}
    \lambda_0^* = 1
\end{equation}
\textbf{\textit{Dimostrazione -}} Si supponga per assurdo $\lambda_0^* = 0$. Allora nella funzione lagrangiana sparirebbe il termine dipendente dal funzionale di costo $J(z)$ e la prima condizione delle (8) diventerebbe
\begin{equation}
    0^T = \frac{\partial L}{\partial z}\big\rvert^* = 0 + \lambda^{*T} \frac{\partial h(z)}{\partial z} + \eta^{*T} \frac{\partial g(z)}{\partial z}
\end{equation} 
Che in forma vettoriale diventa
\begin{equation}
    0^T = 
    \begin{pmatrix}
        \lambda^{*T} & \eta^{*T}
    \end{pmatrix}
    \begin{pmatrix}
        \frac{\partial h(z)}{\partial z} \\
        \frac{\partial g(z)}{\partial z}
    \end{pmatrix}
\end{equation}
Dove l'ultimo vettore è proprio la matrice jacobiana dei vincoli attivi. Per ipotesi $z^*$ è un punto di regolarità dei vincoli e quindi tale matrice ha rango pieno. Dunque l'unica possibilità per cui il prodotto con i moltiplicatori valga $0^T$ è che i moltiplicatori siano tutti nulli, il che è assurdo perchè contraddice l'ipotesi. Pertanto, siccome $\lambda_0^* \neq 0$, sarà sempre possibile dividere tutti i moltiplicatori per $\lambda_0^*$, ottenendo così dei nuovi moltiplicatori $\bar{\lambda_0^*}, \bar{\lambda^*}, \bar{\eta^*}$, con $\bar{\lambda_0^*} = 1$
\nobreak 
\hfill
$\square$
Le due condizioni sono dei precedenti teoremi sono necessarie e quindi non danno direttamente luogo a dei punti di minimo, ma a dei punti candidati ad essere di minimo, che vengono definiti estremali. Gli estremali possono essere 
\begin{itemize}
    \item normali, se $\lambda_0^* \neq 0$ (e quindi conseguentemente $\lambda_0^* = 1$)
    \item non normali, se $\lambda_0^* = 0$ 
\end{itemize} 
Il teorema (2.7) offre delle condizioni meno stringenti rispetto al (2.8), che fissando $\lambda_0^* = 1$ esclude gli estremali non normali, e pertanto fornisce un numero maggiore di candidati.
\medskip
\newline
\textbf{\textit{Esempio 2.1}} 
\nobreak 
\hfill
$\square$
\medskip
\newline
Le condizioni necessarie appena viste diventano anche sufficienti nel caso in cui il funzionale di costo $J(z)$ e l'insieme ammissibile $\mathcal{D}$ siano convessi. La particolarizzazione dell'insieme di ammissibilità al caso convesso può essere effettuata sulla base del seguente lemma
\medskip
\newline
\textbf{\textit{Lemma 2.13 -}} Sia $\mathcal{D}$ un insieme definito da vincoli di uguaglianza $h(z) = 0$ e di disuguaglianza $g(z) \leq 0$. Se 
\begin{itemize}
    \item $g(z)$ è dato da tutte funzioni convesse
    \item $h(z)$ è dato da funzioni lineari (o affini) del tipo $h_j(z) = c_j^Tz_j + b_j$, $j = 1,2,\dots,\mu$
\end{itemize}
Allora $\mathcal{D}$ è convesso.
\nobreak 
\hfill
$\square$
\medskip
\newline
\textbf{\textit{Teorema 2.14 -}} In riferimento al problema di minimo vincolato con un funzionale di costo $J(z)$ convesso e con opportuni vincoli $h(z)$ affini e $g(z)$ convessi, di modo che l'insieme $\mathcal{D}$ sia convesso. Se $z^0$ è un punto ammissibile ed esistono moltiplicatori $\lambda^0$ e $\eta^0$ tali che valgano
\begin{equation}
\begin{cases}
    \frac{dJ}{dz} + \lambda^{0T} \frac{dh}{dz} + \eta^{0T} \frac{dg}{gz} = 0^T \\
    \eta_i^0 g_i(z^0) = 0 \quad i = 1,2,\dots,\sigma_a \\
    \eta_i^0 \geq 0 \quad i = 1,2,\dots,\sigma_a
\end{cases}
\end{equation} 
Allora $z^0$ è un punto di minimo globale. Inoltre, se $J$ è strettamente convessa, $z^0$ è l'unico punto di minimo globale.
\medskip
\newline
\textbf{\textit{Dimostrazione -}} Posto $z$ un qualunque punto ammissibile, allora valgono i vincoli e quindi si può scrivere
\begin{equation}
    J(z) \geq J(z) + \underset{=0}{\underbrace{\lambda^{0T}h(z)}} + \underset{\leq 0}{\underbrace{\eta^{0T}g(z)}}
\end{equation}
Siccome per ipotesi le funzioni $J$ e $g$ sono convesse e le funzioni $h$ sono affini, allora si ha
\begin{equation}
    J(z) \geq J(z^0) + \frac{dJ}{dz}(z-z^0) + \lambda^{0T}\big[ h(z^0) + \frac{dh}{dz}(z-z^0) \big] + \eta^{0T}\big[ g(z^0) + \frac{dg}{dz}(z-z^0) \big] 
\end{equation}
Per le ipotesi del teorema, nel punto $z^0$ si ha $\eta^{0T}g(z^0) = 0$, inoltre poichè, sempre per ipotesi, $z^0$ è un punto ammissibile, allora valgono i vincoli, ovvero $h(z^0) = 0$. Quindi raccogliendo $(z-z^0)$ si ottiene
\begin{equation}
    J(z) \geq J(z^0) + \underset{=0^T}{ \underbrace{ \bigg[ \frac{dJ}{dz} + \lambda^{0T} \frac{dh}{dz} + \eta^{0T} \frac{dg}{dz} \bigg] }} (z-z^0) 
\end{equation}
E di nuovo, per le ipotesi (la prima delle (12)), si può scrivere
\begin{equation}
    J(z) \geq J(z^0)
\end{equation}
Ovvero $z^0$ è un punto di minimo globale. Se inoltre $J$ fosse strettamente convessa, allora le stesse relazioni varrebbero strettamente, giungendo quindi a 
\begin{equation}
    J(z) > J(z^0)
\end{equation}
Che implica l'unicità del minimo.
\nobreak 
\hfill
$\square$
\end{document}
